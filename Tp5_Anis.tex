\documentclass[a4paper,reqno,11pt]{article}
\def\Annee{2018-2019}
\def\NumeroTP{2}
\def\Theme{Estimation ponctuelle et par intervalle de confiance}

\usepackage[utf8]{inputenc} 
\renewcommand*\familydefault{\sfdefault} 

\input ListeDefinitions_TD

\begin{document}

\noindent
{\large M.A. Haddouche \\ A. Rogozan \hfill Année 2019-2020} 
\vspace{4ex}

\begin{center}
{\Large ASI3 } \\
\vspace{1ex}

{\bf\large 

\  \Theme}
\end{center}
\vspace{1ex}


\begin{partie}{\sf \ 
		\vspace{1ex} 
		
L'objet de cette partie est d'étudier par simulations
le comportement de l'estimateur du maximum de vraisemblance
du paramètre d'une loi géométrique.
On rappelle que la loi géométrique de paramètre
$p$ est la loi du nombre $X$ d'épreuves de Bernoulli
indépendantes de même paramètre $p$,
nécessaires à l'obtention d'un succès.
On note alors $X\sim G(p)$.
Ainsi, $X$ prend les valeurs 1, 2, 3, \ldots avec
pour tout $x\in \bkN^*$, \
$\bkP\cro{X = x} = p (1-p)^{x-1}$.
En utilisant la fonction génératrice, on montre
que $\bkE(X) = {1}/{p}$.

%%%%%%%%%%%%

		
		Soit $X$ une variable aléatoire 
		de loi géométrique de paramètre $p$ avec $p \in ]0,1[$.
		\noindent
		On s'intéresse à l'estimation du paramètre $p$.
		On sait que $\bkE(X) = {1}/{p}$.
		Il est donc facile  de proposer un estimateur du paramètre $p$.
		En effet, si on dispose d'un échantillon
		de variables aléatoires $X_1, X_2, \ldots, X_n$, indépendantes et de même loi
		géométrique de paramètre $p$, alors
		$\Xbar_n$ est un estimateur sans biais de ${1}/{p}$.
		Ainsi, pour estimer $p$, on peut proposer l'estimateur
		$\pchap = {1}/{\Xbar_n}$.
		Il se trouve que cet estimateur est aussi 
		l'estimateur du maximum de vraisemblance.
		L'objet de cette première partie est d'étudier par simulations le comportement à distance finie de l'estimateur du paramètre $p$.
		
			\vspace{4ex}
			
		\begin{itemize}
			\item[1] Simuler 400 échantillon de 1000 réalisations d'une variable aléatoire géométrique de paramètre $p=2/5$ et stocker la dans la matrice \verb|data|. 
			
			\vspace{1ex}
						
			\item[2] Pour chaque colonne de données, construisez la suite des valeurs \textbf{successives} de l'estimateur du paramètre $p$ et dont on stocker les résultats dans une matrice \verb|M|. 
			
			\vspace{1ex}
			
			\item[3] Tracer, sur le même graphique, les boites à moustaches des 400 estimations pour les tailles d'échantillon $n=50, 100,$$200, 500, 1000$ et tracer, sur ce mème graphique, la valeur de $p$.
			
			\vspace{1ex}
			
			\item [4] Analyser et commentez les résultats obtenus. 
			Que sommes-nous amener à illustrer avec un tel graphique ?
			
			
%			\vspace{1ex}
%						
%			\item [5]
%			Faite tourner votre programme pour d'autres valeurs du paramètre $p$
%			(on pourra essayer les valeurs $p=0.1$, $p=0.7$ et $p=0.9$.
%			Observe-t-on un changement de comportement de l'estimateur en fonction de la valeur de $p$ ?
		\end{itemize}
	
	
		
		
	
}\end{partie}
%%%%%%%%%%%%
\begin{partie}{\sf \ 
		\vspace{1ex}
		
Dans un centre avicole, des études antérieures 
ont montré que le poids d’un oeuf choisi au hasard 
peut être considérée comme la réalisation d’une variable aléatoire gaussienne  $X$, 
d'espérance  $\mu$ et de variance $\sigma^2$. 
On admet que les poids des oeufs sont indépendants les uns des autres.
On prend un échantillon de $n = 36$ oeufs que l’on pèse.
Les mesures obtenues (exprimées en $g$) sont dans (par ordre croissant) le fichier {\verb oeufs.txt } sur Moodle 
%
\begin{itemize}
	\item[1] Stocker les données dans le vecteur \verb|x|. (utiliser la fonction \verb|np.genfromtxt(,)|) et tracer l'histogramme des fréquence et superposer sur ce dernier la densité de la loi normal. Commenter
	
	\vspace{1ex}
	
	\item[2] Proposer des estimateurs sans biais de la moyenne $\mu$  et de la variance $\sigma^2$  et donner une estimation de ces deux paramètres. On notera, respectivement, ces  estimateur $\bar{X}, S^2$ et les estimations  $\bar{x}$ et $s^2$
	
%	\vspace{1ex}
%	\item[3]  Construire un intervalle de confiance au niveau 95\% pour le poids  moyen des oeufs.

\end{itemize}	

%Comme nous sommes  dans le cadre de l'échantillon gaussien, et que $\sigma$ est inconnue,
%l'intervalle de confiance au niveau 95\%
%du poids moyen des oeufs est donné par :
%
%$$\mbox{IC}_{95\%}(\mu) = \cro{m \pm t_{35, 0.975} \,\dfrac{s}{\sqrt{36}}} $$
%
%où $t_{35, 0.975}$ désigne le quantile d'ordre 0,975 d'une Student à 35 ddl.


	\vspace{1ex}
\begin{itemize}
	 
	 \item[3] Quelle est la loi de la variable $\sqrt{n}(\bar{X}-\mu)/S$.
	\vspace{1ex}
	
	\item [4]  Construisez un intervalle de confiance  bilatérale au niveau 95\% pour le poids  moyen des oeufs et commenter.
\end{itemize}

	\vspace{1ex}
	
On souhaiterais maintenant construire un intervalle de confiance bilatérale au niveau 95\% pour la variance $\sigma^2$ de la variable aléatoire $X$

\vspace{1ex}

\begin{itemize}
	
	\item[5] Quelle est la loi de la variable aléatoire $Z=(n-1)S^2 / \sigma^2$, où $S^2$ est un estimateurs sans biais de $\sigma^2$
	
	\item[6] Calculez les quantiles d'ordre $0.975$ et $0.025$ de $Z$.
	
	\vspace{1ex}
	
	\item[7] Construire un intervalle de confiance au niveau 95\% pour la variance et commenter.



\vspace{1ex}

\item[7] On sait que la demi-longueur
de l'intervalle de confiance pour $\mu$ est égale à 
$t_{35, 1-\alpha/2}\times  s / \sqrt{36}$. A quel niveau de confiance correspondrait un intervalle centré en $m$ et de demi-longueur 0,76 ?
(\verb|Utiliser la fonction chi2.cdf() dans scipy.stats|)  

\end{itemize}
}\end{partie}




\end{document}